\chapter{Appendices}

\section{GitHub Repository \& Web Application URLs}
\begin{itemize}
    \item GitHub Repository: 
    \url{https://github.com/johnshields/Repota-App}
    \item Repota Web Application: 
    \url{https://www.repota-service.com}
\end{itemize}

\section{How to Install and Run}
\begin{itemize}
\item All changes (besides MySQL details, the database, and 3rd Party API config files) have been made for anyone to run locally.
\end{itemize}

\begin{itemize}
  \item Requirements
    \begin{itemize}
    \item Git = \url{https://git-scm.com/downloads}
    \item NPM = \url{https://www.npmjs.com/get-npm}
    \item Golang - 1.15.3 = \url{https://golang.org/dl/}
    \item MySQL = \url{https://dev.mysql.com/downloads/shell/}
    \end{itemize}
\end{itemize}

\subsubsection{Clone the GitHub Repository}
\begin{itemize}
  \item Open a directory of your choice in Command-Line and enter:
    \begin{itemize}
    \item git clone https://github.com/johnshields/Repota-App.git
    \end{itemize}
\end{itemize}

\newpage
\subsubsection{Repota Front-end}
\begin{itemize}
  \item Open the repository directory in Command-Line and enter:
    \begin{itemize}
    \item cd repota/repotaApp
    \item npm install @angular/cli
    \item npm install
    \item npm run build
    \item ionic serve
    \begin{itemize}
        \item Running on Localhost: \url{http://localhost:8100}
    \end{itemize}
    \end{itemize}
\end{itemize}

\subsubsection{Horton Back-end}
\begin{itemize}
  \item Ensure the following have been fulfilled:
    \begin{itemize}
    \item Create the Database in a MySQL Console.
        \begin{itemize}
            \item Location - \url{/database/REPOTA_DB.sql}
        \end{itemize}
    \item Edit the config.ini file with your MySQL details.
        \begin{itemize}
            \item Location - \url{/horton/go/config/config.ini}
        \end{itemize}
    \item Acquire an App ID and API key from Back4App to use their service. (First 10k requests are free).
        \begin{itemize}
            \item URL - \url{https://www.back4app.com/}
        \end{itemize}
    \item Add in the App ID and API key into the .txt config files.
        \begin{itemize}
            \item Location - \url{/horton/go/config/}
        \end{itemize}
    \end{itemize}
\end{itemize}

\begin{itemize}
  \item Open the repository directory in Command-Line and enter:
    \begin{itemize}
    \item cd horton
    \item go mod download
    \item go build \&\& go run main.go
    \begin{itemize}
        \item Running on Localhost: \url{http://localhost:8080/api/v1}
    \end{itemize}
    \end{itemize}
\end{itemize}
