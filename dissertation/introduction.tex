\chapter{Introduction}

\section{Abstract}
Service reports are essential records for company technicians in the equality of agricultural machinery, automobiles, computers, air-crafts, and military machines. For instance, a history of these reports needs to be stored for easy access as problems may arise with the exact automobiles recorded in the report. Having this information outlines previous problems and allows to identify the problem should it be related to the initial.

\section{Acknowledgements}
 I would like to thank my lectures throughout the years and my supervisor, Andrew Beatty, for his advice and for pointing me in the right direction throughout the year. Lastly, I would like to thank my brother for inspiring the idea that turned into a successful final product.

\section{Project Introduction}
In an achievable manner and at a Bachelor of Honours standard, I aimed to develop a service report, SaaS (Software as a Service) application for automobile technicians titled 'Repota' for the \textit{Applied Project and Minor Dissertation}. I decided to complete this application after observing my brother, who is employed in a workplace that specializes in the supply and service of new and used Mobile Plant and Equipment, have difficulties with documenting reports in a timely fashion. The reports are currently completed on paper or with Microsoft Excel. I decided to create an application to write up service reports that is productive, easy to use and effective. After careful consideration, I decided the application's features would consist of a user account system, CRUD (Create, Read, Update and Delete) operations for the reports and an option to export them to PDFs.
\\\\ This application is at Level 8 standard as it aimed to be at a full-stack level; it had to contain three components: a front-end, back-end, and a database. The technologies to develop such an application to be Angular and Ionic for the front-end, OpenAPI to design the front-end and back-end APIs (Application Programming Interface), Golang for the back-end and MySQL for the database. The entire application would then be hosted through AWS (Amazon Web Services). Entailing S3 Bucket (Simple Storage Service) for the front-end, Elastic Beanstalk for the back-end and an EC2 (Elastic Compute Cloud) Ubuntu Virtual Machine for the database.
\\\\ The purpose of this dissertation is to unpack the planning, implementation and reflection of the technologies of the application and the objectives achieved, how it was designed and how it all came to together as a whole. It will firstly look at the methodology of the project's development. Secondly, it will investigate the technologies that were researched and implemented throughout the project. Following on, it will demonstrate the designs and evaluation of the project. The conclusion will then summarise and reflect on the achievements of the project's main objectives.

\begin{figure}[H]
    \caption{Application Logo}
    \label{image:RepotaLogo}
    \centering
    \includegraphics[width=0.6\textwidth]{images/repota/repotaApp_logo.png}
\end{figure}

\newpage
\subsection{Objectives}
The overall aim of the project was to achieve the following objectives: 
\begin{itemize}
    \item Comprehensive MySQL database with all the required information.
     \begin{itemize} 
        \item Users and Service reports.
    \end{itemize}
    \item OpenAPI specification of Front-end and Back-end APIs.
    \item Robust and RESTful back-end connected to the database in Golang.
    \begin{itemize} 
        \item CRUD Operations for service reports.
        \item Account system for users; register, login and logout.
        \begin{itemize} 
            \item Microservices - Sessions and Cookies for users.
        \end{itemize}
        \item 3rd Party API access for vehicle information.
    \end{itemize}
    \item User friendly front-end with the Angular and Ionic Framework connected to the back-end.
    \begin{itemize} 
        \item Report - CRUD operation pages.
        \item Account - Register, Login and Logout pages.
        \item Option to export reports to PDFs.
        \item Front-end UI responsiveness for multiple devices.
    \end{itemize}
    \item Testing of back-end and front-end.
    \begin{itemize}
        \item Suite of integration tests for the back-end's functionality.
        \item Suite of high level behaviour tests for the Front-end.
    \end{itemize}
    \item Database, Back-end and Front-end hosted on AWS.
    \begin{itemize} 
        \item Database hosted on a EC2 Ubuntu Virtual Machine.
        \item Back-end hosted with Elastic Beanstalk.
        \item Front-end hosted with S3 Bucket.
        \item HTTPS for the hosted back-end and front-end.
    \end{itemize}
\end{itemize}

\newpage
\section{Resources URLs}
\begin{itemize}
    \item GitHub Repository: 
    \url{https://github.com/johnshields/Repota-App}
    \item Repota Web Application: 
    \url{https://www.repota-service.com}
    \item Video Demonstration: 
    \url{video_link_here}
\end{itemize}

\subsubsection{Repository Contents}
\begin{itemize}
  \item Repota
    \begin{itemize}
    \item Source code of the Front-end.
    \end{itemize}
  \item Horton
    \begin{itemize}
    \item Source code of the Back-end.
    \end{itemize}
  \item API Specification
    \begin{itemize}
    \item OpenAPI Specification of the Front-end and Back-end APIs.
    \end{itemize}
  \item Database
    \begin{itemize}
    \item Script of the Application's Database.
    \end{itemize}
  \item Workings
    \begin{itemize}
    \item Miscellaneous files for rough work. 
    \end{itemize}
\end{itemize}

\newpage
\section{Chapter Descriptions}
\subsubsection{Methodology}
Methodology discusses the project's scope, its goal, and any other ideas that were considered. Along with the development methodology used for the project's development approach, its time management, and the initial setup for verification and testing.

\subsubsection{Technology Review}
The Technology Review discusses the technologies researched and implemented into the project. Along with describing how they fitted the requirements of the project.

\subsubsection{System Design}
System Design goes through the entire design of the project and how it was all constructed together to become a final product.

\subsubsection{System Evaluation}
System Evaluation tells how the final product was thoroughly tested through out the front-end and back-end. Including a section for issues encountered that caused long pauses in development.

\subsubsection{Conclusion}
The Conclusion summaries the project, possible next steps for it, what was learned from the project and addresses the objectives outlined in the Introduction. 
