\chapter{Conclusion}
Repota turned out to be a great success. It has all its intended functionality. I aimed to develop a service report application for automobile technicians using MySQL, OpenAPI, Go, Angular and Ionic. The finalized Repota provides the service of creating, storing, editing, and deleting of service reports. I believe the idea of Repota could one day be very successful. If I am to continue with this idea, I would definitely continue using MySQL, OpenAPI, and Go. Perhaps I would change the front-end technologies to the likes of Rust or Flutter as these JavaScript/TypeScript based frameworks can become obsolete relatively quickly since there is always a new "big" one every 1-2 years.  
\\\\ Overall, I am very happy with the final product and what I accomplished. I think the scope was a perfect size for a one-person project, resulting in a satisfactory output from hard work, problem-solving, and improvisation. While having slow initial compile times, Angular and Ionic being familiar frameworks made the development of Repota a somewhat satisfying process. Swagger turned out to be a surprisingly easy tool for a first-time user and is great for creating an OpenAPI specification. Learning Go for the first time was a welcome challenge that resulted in a pleasing outcome for Horton and proved to be a great language to learn as it is very fast and quite object-oriented. The combination of MySQL, Go and Angular, and Ionic worked out quite well.
\\\\ Independently learning a new language has improved my skills as a developer immensely. Before this year, I was getting through by the skin of my teeth. In the third year of this course, I had to repeat a module for the first semester. This setback helped me mature a lot about college and prepare my head for this project. With learning Go came a lot of debugging since I was uncomfortable with it. However, through this learning process, I have become better at other languages used throughout this course. From learning Go, I have a better understanding of programming itself as I can now formulate solutions that could possibly lead to something successful rather than having no idea where to begin with, which I suffered a lot from throughout the past years in the course.
\\\\ Having said all that, I am very proud of myself for getting to the other side, and I would like to thank my lectures and my supervisor, Andrew Beatty, for his advice and for pointing me in the right direction throughout the year. Lastly, I would like to thank my brother for giving me the inspiration that turned into a successful final product.