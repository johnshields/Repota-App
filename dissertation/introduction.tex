\chapter{Introduction}

\section{Project Introduction}
In an achievable manner and at a Bachelor of Honors standard, I aimed to develop a service report application for automobile technicians titled 'Repota' for the \textbf{\textit{Applied Project and Minor Dissertation}}. The inspiration for this application came from my brother, who is employed in a workplace that specializes in the supply and service of new and used Mobile Plant and Equipment. Their reports are generally carried out on paper or with Microsoft Excel. The application's features would have to consist of a user account system, CRUD (Create, Read, Update and Delete) operations for the reports, and an option to export them to PDFs. The application's production aimed to be at a full-stack level. Meaning the application would have to contain three components. A front-end, back-end, and a database. The technologies to develop such an application to be Angular and Ionic for the front-end, OpenAPI to design the front-end and back-end APIs (Application Programming Interface), Golang for the back-end, and MySQL for the database. The entire application would then be hosted through AWS (Amazon Web Services). Entailing S3 Bucket (Simple Storage Service) for the front-end, Elastic Beanstalk for the back-end, and an EC2 (Elastic Compute Cloud) Ubuntu Virtual Machine for the database. Throughout this dissertation I will be going through the technologies of the application, how it was designed and how it all came to together as a whole. 

\begin{figure}[t]
    \caption{Application Logo}
    \label{image:RepotaLogo}
    \centering
    \includegraphics[width=0.6\textwidth]{images/repota/repotaApp_logo.png}
\end{figure}
\newpage
\section{Resources URLs}
\begin{itemize}
    \item GitHub Repository: 
    \url{https://github.com/johnshields/Repota-App}
    \item Repota Web Application: 
    \url{https://www.repota-service.com}
    \item Video Demonstration: 
    \url{video_link_here}
\end{itemize}

\subsubsection{Repository Contents}
\begin{itemize}
  \item Repota
    \begin{itemize}
    \item Source code of the Front-end.
    \end{itemize}
  \item Horton
    \begin{itemize}
    \item Source code of the Back-end.
    \end{itemize}
  \item API Specification
    \begin{itemize}
    \item OpenAPI Specification of the Application.
    \end{itemize}
  \item Database
    \begin{itemize}
    \item Script of the Application's Database.
    \end{itemize}
  \item Workings
    \begin{itemize}
    \item Miscellaneous files for rough work. 
    \end{itemize}
\end{itemize}

\newpage
\section{Chapter Descriptions}
\subsubsection{Methodology}
Methodology discusses the project's scope, its goal, and any other ideas that were considered. Along with information on service reports, the development methodology used for the project's development approach, its time management, and an initial setup for verification and testing.

\subsubsection{Technology Review}
The Technology Review discusses the technologies researched and implemented into the project. Along with describing how they fitted the requirements of the project.

\subsubsection{System Design}
System Design goes through the entire design of the project and how it was all constructed together to become a final product.

\subsubsection{System Evaluation}
System Evaluation tells how the final product was thoroughly tested through out the front-end and back-end. Including a section for issues encountered that caused long pauses in development.
