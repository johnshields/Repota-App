\chapter{Introduction}
\section{Repository Links}
\begin{itemize}
    % Dissertation Repository
    \item Dissertation (Repository): \ur{https://github.com/johnshields/AP-MD-FYP/tree/main/dissertation} 
     % Application Repository
    \item Repota (Repository): \url{https://github.com/johnshields/AP-MD-FYP/tree/main/Repota}
        % Back-end Server Repository
    \item Horton (Repository): \ur{https://github.com/johnshields/AP-MD-FYP/tree/main/horton} 
    % Database Repository
    \item Database (Repository): \ur{https://github.com/johnshields/AP-MD-FYP/tree/main/database} 
\end{itemize}

\section{Project Introduction}
The purpose of this project was to develop a service report application that relied on modern day frameworks and technologies. The project was focused at using these frameworks and technologies to build an app that is designed for technicians to fill out a service report for auto dealerships and rental companies. With this app technicians can fill out a service report on a recent job they did. This is done in a step by step process. Technicians can not only create their reports they can view, edit, delete and export them to a PDF. These reports are usually done on paper or excel. I did find some similar apps that have a similar idea but it is almost as if they are trying to do too much for what they are supposed to be.  With that said there is no app that does just this service. I believe this app is a necessity for technicians for their day-to-day work.
\\\\ The aim of this project was to amalgamate a  sophisticated front-end framework to a quick back-end which would be connected to a stable and strong database. These technologies choices that were researched and implemented are as follows: Angular and Ionic for the front-end, Golang for the back-end and MySQL for the database. Before this project Golang was completely new to me. I wanted to use Golang as it is a very low-level language which was just what I needed for getting data from the database and displaying it to the front-end.
\\\\

\begin{figure}[h!]
    \caption{Application Icon} % app logo
    \label{image:repotaChatLogo}
    \centering
    \includegraphics[width=0.8\textwidth]{images/repotaApp_logo.png}
\end{figure}